\chapter{Исследование и построение решения задачи}
\label{sec:Chapter3} \index{Chapter3}

\section{Анализ функции \texttt{S\_regmatch}}

Анализ приведенной в главе \ref{sec:Chapter1} функции \texttt{S\_regmatch} из бенчмарка\\\texttt{perlbench\_r} который является частью набора бенчмарка SPEC CPU\textsuperscript{\tiny\textregistered} 2017, показал:
\begin{itemize}
    \item Вследствие внутренней структуры функции, во вложенных циклах, образованных безусловными переходами внутри некоторых блоков \texttt{case} конструкции \texttt{switch} используется большое количество значений, являющихся инвариантами для внутреннего цикла, но не являющихся инвариантами для внешнего цикла.
    \item Все инструкции порождающие такие значения, выносятся в заголовок внешнего цикла в ходе выноса инвариантов из циклов, так как он является предзаголовком для внутренних циклов и эти значения используются внутри тела внутреннего цикла, а следовательно не могут быть пропагированы в блоки выхода.
    \item Анализ частоты исполнения базовых блоков показывает, что частота исполнения заголовка внешнего цикла много больше частоты исполнения блоков внутренних циклов, и для многих инструкций, упомянутых выше, эффективна пропагация в тело внутреннего цикла.
        Это следует из контекста применения данной функции.
        \texttt{S\_regmatch} является функцией проверки на совпадение строки и скомпилированного регулярного выражения.
        Все обсуждаемые выше внутренние циклы являются обработкой особых токенов, таких как токен предпросмотра.
        Таким образом, можно проследить, что такие токены редки в скомпилированном выражении, т.е. $p_i$ - вероятность перейти из конструкции \texttt{switch} в блок \texttt{case i}, отвечающий внутреннему циклу, и число итераций внутреннего цикла $c_i$ - величина порядка единицы.
        Число вариантов в конструкции \texttt{switch}, напротив, велико, а частота исполнения блока цикла не больше частоты исполнения заголовка цикла, при условии отсутствия внутренних циклов, которую можно оценить как:
        $$ f_i = f_o  p_i^{c_i}  c_i $$
        Где $f_o$ - частота заголовка внешнего цикла, в котором до пропагации инвариантов в тело цикла находятся обсуждаемые инструкции.
        Таким образом, условие при котором, пропагация в заголовок внутреннего цикла будет эффективна:
        $$ p_i^{c_i}  c_i \leq 1 $$
        Из структуры неравенства видно, что оно будет часто выполняться, при принятых условиях на $p_i$ и $c_i$.
        При наличии в теле внутреннего цикла ветвлений и условии использования инварианта только на некоторых путях, аналогичное выражение принимает вид:
        $$ p_i^{c_i} c_i \sum_j{w_j} \leq 1 , \sum_j{w_j} < 1 $$
        Где $w_j$ - отношение частоты исполнения блока $j$ к частоте исполнения заголовка внутреннего цикла.
    \item В ходе пропагации инвариантов в тело цикла, многие такие инструкции не меняют своего расположения, несмотря на то, что для этих значений пропагация будет выгодна с точки зрения минимизации суммарной частоты исполнения блоков расположения инварианта.
\end{itemize}

Из сказанного выше следует, что разрыв в производительности с компилятором GCC, обусловлен субоптимальным алгоритмом пропагации инвариантов в тело цикла.
Поэтому задача улучшения алгоритма расположения инвариантов цикла сводится к улучшению именно алгоритма пропагации инвариантов в тело цикла.

\subsection{Упрощенный пример \texttt{S\_regmatch\_draft}}

Для демонстрации рассмотрим более простую функцию, обладающую схожими свойствами.
Листинг кода этой функции представлен в приложении \ref{lst:S_regmatch_draft_c}.
Упрощенная визуализация графа потока управления функции \texttt{S\_regmatch\_draft} представлена на рисунке \ref{fig:S_regmatch_darft}.

Эта функция предназначена для нахождения первого вхождения шаблона \texttt{pattern} в строке \texttt{string}.
Возвращаемое значение -- флаг, найден ли шаблон или нет.
Позиция начала подстроки, совпадающей с шаблоном и её длина возвращаются через указатели \texttt{pos} и \texttt{len} соответственно.
В шаблоне, помимо символов, могут встречаться специальные токены \texttt{LA\_STOP} и \texttt{LA\_START}.
Часть шаблона, заключенная между этими токенами - окно предпросмотра.
Подстрока совпадает с шаблоном, оканчивающимся окном предросмотра, только если подстрока совпадает с частью шаблона, предшествующей окну предпросмотра, и продолжение подстроки совпадает с окном предпросмотра.

\begin{figure}
    \centering
    \includegraphics[width=\linewidth]{images/S_regmatch_draft.pdf}
    \caption{Схема потока упровления в функции \texttt{S\_regmatch\_draft}}
    \label{fig:S_regmatch_darft}
\end{figure}

Здесь можно выделить:
\begin{itemize}
    \item Внешний цикл:

        \texttt{reenter\_switch} $\rightarrow$ \texttt{default} $\rightarrow$ \texttt{while} $\rightarrow$ \texttt{reenter\_switch}

    \item Внутренние циклы:

        \begin{enumerate}
            \item \texttt{reenter\_switch} $\rightarrow$ \texttt{case LA\_START} $\rightarrow$ \texttt{LA\_START\_then} $\rightarrow$\\ $\rightarrow$ \texttt{reenter\_switch}
            \item \texttt{reenter\_switch} $\rightarrow$ \texttt{case LA\_START} $\rightarrow$ \texttt{reenter\_switch}
            \item \texttt{reenter\_switch} $\rightarrow$ \texttt{case LA\_STOP} $\rightarrow$ \texttt{LA\_STOP\_then} $\rightarrow$\\$\rightarrow$ \texttt{reenter\_switch}
            \item \texttt{reenter\_switch} $\rightarrow$ \texttt{case LA\_STOP} $\rightarrow$ \texttt{reenter\_switch}
        \end{enumerate}
\end{itemize}

При более детальном рассотрении этих циклов, можно заметить, что инварианты этих циклов обладают схожими свойствами с инвариантами внутренних циклов функции \texttt{S\_regmatch}.
Так, например, символ \texttt{string[i]}, как и сама переменная \texttt{i}, индуктивная для внешнего цикла, являются инвариантами для внутренних циклов.

Рассмотрим построение упрощенной формы циклов (англ. Loop Simplify) для этого графа.
Схема графа потока управления функции после приведения внутреннего цикла 2 к упрощенной форме представлена на рисунке \ref{fig:S_regmatch_darft_loop_simplify_one}.
В ходе преобразования был выделен предзаголовок цикла 2 \texttt{do.body} и заголовок \texttt{loop2.header}.

\begin{figure}
    \centering
    \includegraphics[width=\linewidth]{images/S_regmatch_draft_after_loop_simplify_one.pdf}
    \caption{Построение упрощенной формы циклов в функцие \texttt{S\_regmatch\_draft}}
    \label{fig:S_regmatch_darft_loop_simplify_one}
\end{figure}

После завершения построения упрощенной формы циклов для всех циклов функции, граф потока управления будет иметь вид, представленый на рисунке \ref{fig:S_regmatch_darft_loop_simplify_all}.

\begin{figure}
    \centering
    \includegraphics[width=\linewidth]{images/S_regmatch_draft_after_loop_simplify_all.pdf}
    \caption{Упрощенная форма циклов функции \texttt{S\_regmatch\_draft}}
    \label{fig:S_regmatch_darft_loop_simplify_all}
\end{figure}

Из рисунка видно, что после построения упрощенной формы циклов, каждый из вложенных циклов обладает:
\begin{itemize}
    \item Заголовком (\texttt{loop1-3.header} и \texttt{reenter\_switch} для циклов 1-4 соответственно).
    \item Предзаголовком (\texttt{do.body} и \texttt{loop1-3.header} для циклов 1-4 соответственно).
\end{itemize}
Таким образом форма была успешно построена и в функцие нет несократимого потока управления.

Рассмотрим теперь расположение инварианта \texttt{string[i]}.
После выноса инвариантов из цикла, он будет расположен в предзаголовке цикла 1 \texttt{do.body}.
Если этот инвариант не будет пропагирован в тела внутренних циклов, число исполнений интрукций пораждающих этот инвариант за один вход в функцию совпадет с числом итераций внешнего цикла, а именно с длинной строки \texttt{string}.
Если же он будет пропагирован в базовые блоки \texttt{LA\_START\_then} и \texttt{LA\_STOP\_then}, анализ исходного кода может показать, что соответствующее число исполнений составит сумму числа токенов \texttt{LA\_START} и \texttt{LA\_STOP} в шаблоне \texttt{pattern}.
Из контекста применения функции очевидно, что длина шаблонна, а следовательно и сумма числа токенов \texttt{LA\_START} и \texttt{LA\_STOP} в шаблоне, меньше, чем длина строки, в которой ищется совпадающая с шаблоном подстрока.
Таким образом, расположение инварианта в блоках \texttt{LA\_START\_then} и \texttt{LA\_STOP\_then} будет оптимальным с точки зрения минимизации числа исполняемых инструкций.

\section{Улучшения алгоритма пропогации инвариантов в тело цикла}

При детальном рассмотрении используемого алгоритма пропагации были замечены две возможности для улучшения:

\subsection{Пропагация инвариантов с использованием в \texorpdfstring{$\varphi$}{phi} узлах}

Многие инварианты цикла могут быть использованы в некотором $\varphi$ узле.
Пример такого цикла приведен на рисунке \ref{fig:cfg_phi}.
В таком случае, невозможно переместить инвариант в базовый блок в котором находится этот $\varphi$ узел.
Следует рассматривать как кандидата для пропагации не базовый блок содержащий сам узел, а базовый блок из которого поток управления входит в узел с данным значением.
Это допустимо, так как использование значения инварианта происходит не в самом базовом блоке, в котором находится $\varphi$ узел, а на ведущем в него ребре.
Соответственно, если определение значения происходит в базовом блоке, из которого выходит ребро, значение инварианта определено к моменту входа в $\varphi$ узел и цепь определение-использование не нарушена.

\begin{figure}
    \centering
    \includegraphics[width=\linewidth]{images/cfg_phi.pdf}
    \caption{Пример графа потока управления цикла с использованием инварианта в $\varphi$-узле}
    \label{fig:cfg_phi}
\end{figure}

\subsection{Пропагация инвариантов во все доминируемые базовые блоки}

Некоторые использования инварианта могут находиться вне цикла.
В таком случае можно рассматривать в качестве кандидатов для пропагации объединение множества базовых блоков цикла с базовыми блоками выхода из цикла.
Это допустимо, так как объединение множеств базовых блоков, доминируемых блоками выхода из цикла, совпадает с множеством базовых блоков, доминируемых предзаголовком, за вычетом множества блоков цикла.
Это следует из того, что любой путь из предзаголовка в базовый блок вне цикла, пройдет через некоторый блок выхода.
Такой подход не изменит ассимптотику алгоритма, так как число выходных блоков, можно ограничить сверху как произведение максимального числа выходящих ребер из блока цикла на число блоков цикла.
При этом, число выходящих ребер - константа, определяемая терминатором базового блока.

Такой подход к решению проблемы пропагации инвариантов цикла с использованиями в базовых блоках не принадлежащих циклу является корректным, но не обеспечивает расположения инвариантов, оптимального с точки зрения минимизации оценки суммарной частоты исполнения.

Это можно показать на контрпримере, граф потока управления которого представлен на рисунке \ref{fig:cfg_alldom} в упрощенной форме циклов.
Рассмотрим функцию с циклом, в котором существует ветвление потока управления и лишь одна из ветвей содержит использование инварианта.
Блоки циклов доминируют над подграфом, в котором существует аналогичное ветвление.

\begin{figure}
    \centering
    \includegraphics[width=0.5\linewidth]{images/cfg_alldom.pdf}
    \caption{Контрпример графа потока управления, к пропагации в объединение множеств блоков цикла и блоков выхода}
    \label{fig:cfg_alldom}
\end{figure}

Рассмотрим частоты исполнения блоков функции.

Ввдем следующие обозначения:
\begin{itemize}
    \item $p_{if}$ -- вероятность перехода из блока \texttt{if} в блок \texttt{if.then}.
    \item $p_{loop.if}$ -- вероятность перехода из блока \texttt{loop.if} в блок \texttt{loop.if.then}.
    \item $c$ -- число итераций цикла.
    \item $f_{entry}$ -- частота исполнения предзаголовка цикла \texttt{entry}
    \item $f_{loop.if.then}$ -- частота исполнения блока цикла \texttt{loop.if.then}
    \item $f_{if}$ -- частота исполнения блока выхода цикла \texttt{if}
    \item $f_{if.then}$ -- частота исполнения базового блока \texttt{if.then}
\end{itemize}

Частота предзаголовка цикла принимается равной единице.

$$ f_{entry} = 1 $$

Частота базового блока \texttt{if} равна частоте предзаголовка, так как этот базовый блок является единственным блоком выхода цикла.

$$ f_{if} = f_{entry} = 1 $$

Остальные частоты, введенные выше, выражаются следующим образом:

$$ f_{if.then} = f_{if} p_{if} = p_{if} $$
$$ f_{loop.if.then} = f_{entry} c\; p_{loop.if} = c\; p_{loop.if} $$

Пусть существует инвариант цикла, используемый в базовых блоках \texttt{loop.if.then} и \texttt{if.then}.
Этот инвариант будет находиться в предзаголовке цикла \texttt{entry} после выноса инвариантов из тела цикла, так как он используется в блоке цикла \texttt{loop.if.then}.

Суммарная частота иполнения блоков \texttt{loop.if.then} и \texttt{if.then} составляет:

$$ f_1 = f_{if.then} + f_{loop.if.then} = p_{if} + c\; p_{loop.if} $$

Пусть выполняется следующее условие на вероятности переходов:

$$ p_e + c\; p_l < 1 $$

В таком случае, суммарная частота исполнения базовых блоков в которых расположены инструкции, порождающие инвариант, может быть минимизирована при пропагации инварианта в базовые блоки \texttt{loop.if.then} и \texttt{if.then}.

Однако, если ограничиться при объединением множеств блоков цикла с блоками выхода цикла для рассмотрения как кандидатов в целевые блоки пропагации, пропагация не будет эффективна, так как суммарная частота исполнения блока выхода цикла и любого другого блока будет больше, чем частота исполнения предзаголовка:

$$ f_2 = f_{if} + f = 1 + f > 1 = f_{entry} $$

Таким образом, при условии использования инварианта вне цикла, для минимизации суммарной частоты исполнения базовых блоков расположения инварианта, не достаточно ограничиваться рассмотрением как кандидатов в целевые блоки пропагации объединения множеств блоков цикла с блоками выхода цикла, а следует рассматривать все базовые блоки, доминируемые предзаголовком.

\section{Оценка эффективности предлагаемого алгоритма расположения инвариантов цикла}

Рассмотрим расположение инвариантов цикла в компиляторной инфраструктуре LLVM после применения описанного выше алгоритма, модифицированного предложенным образом, а именно:

\begin{itemize}
    \item В ходе выноса инвариантов из цикла инвариант был:
        \begin{itemize}
            \item Удален, если у него нет использований.
            \item Вынесен в подмножество блоков выхода, если все использования инварианта находятся вне цикла.
            \item Вынесен в предзаголовок цикла, если есть блоки цикла использующие значение инварианта.
        \end{itemize}
    \item В ходе пропагации инвариантов в тело цикла:
        \begin{itemize}
            \item Для цикла было построено множество $C$, состоящее из базовых блоков, доминируемых предзаголовком и имеющих меньшую оценку частоты исполнения.
            \item Для каждого инварианта цикла было построено множество $U$, состоящее из базовых блоков, содержащих его использование.
            \item Для каждого цикла строится множество:
                $$ M : M \subset C, \: \forall \: u \in U \: \exists \: m \in M : m \: dom \: u $$
            \item Оценка суммарной частота базовых блоков множества $M$ минимизруется. 
            \item Инвариант был спропагирован в базовые блоки множества $M$, если оно существует, и итоговая оценка суммарной частоты исполнения оказалась меньше, чем оценка частоты исполнения предзаголовка.
        \end{itemize}
\end{itemize}

Таким образом, для инвариантов цикла существует 4 варианта расположения, которые характеризуются в обозначениях формальной постановки задачи из главы \ref{sec:Chapter1} следующим образом:
\begin{enumerate}
    \item Инвариант удален:
        $$ M = U = \emptyset $$
    \item Инвариант расположен в подмножестве блоков выхода.
    \item Инвариант расположен в предзаголовке цикла $p$.
    \item Инвариант расположен в множестве $M$, полученом в ходе пропогации инвариантов в тело цикла.
\end{enumerate}

Проверим эти варианты расположения на соответствие условиям, изложенным в постановке задачи в главе \ref{sec:Chapter1}.

\begin{itemize}
    \item $M \subset D = \{ \: d : p \: dom \: d \: \}$
        \begin{itemize}
            \item Вариант 1 -- выполняется, так как:
                $$ M = \emptyset \in D $$
            \item Вариант 2 -- выполняется, так как предзаголовок доминирует все блоки выхода.
            \item Вариант 3 -- выполняется, так как:
                $$ p \: dom \: p \: \forall \: p $$
            \item Вариант 4 -- выполняется, так как:
                $$M \subset C \subset D$$
        \end{itemize}
    \item $\forall \: u \in U(i) \: \exists \: m \in M : m \: dom \: u $
        \begin{itemize}
            \item Вариант 1 -- выполняется, так как:
                $$ U = \emptyset $$
            \item Вариант 2 -- выполняется по построению.
            \item Вариант 3 -- выполняется, так как:
                $$ p \: dom \: b $$
                Где $b$ - базовый блок, в котором инвариант был расположен изначально.
            \item Вариант 4 -- выполняется по построению.
        \end{itemize}
    \item $\sum_{m \in M}{f(m)} \to min $
        \begin{itemize}
            \item Вариант 1 -- выполняется, так как сумма равна нулю.
            \item Вариант 2 -- не выполняется, так как можно построить контрпример:
                Пусть терминатор блока выхода является условным переходом, и инвариант используется только в одном из непосредственно доминируемых базовых блоков.
                Тогда, размещение инварианта непосредственно в блоке с использованием будет обладать меньшей частотой исполнения.
            \item Вариант 3 -- выполняется, по построению варианта 4:
            \item Вариант 4 -- выполняется по построению.
        \end{itemize}
\end{itemize}

Таким образом, доказано, что полученное расположение корректно и, за исключением случая выноса в выходные блоки, оптимально.
Однако, инструкция, вынесенная в выходной блок не имеет использований в цикле, и может быть пропагирована в последствии в рамках трансформации, которая будет решать задачу минимизации частоты исполнения инструкции на ациклическом графе, представляя каждый цикл его предзаголовком, из которого поток управления переходит в блоки выхода.
Эта трансформация реализована в компиляторной инфраструктуре LLVM.

Приведенное выше рассуждение доказывает, что предложенный алгоритм обеспечивает оптимальное расположение инвариантов цикла.

\newpage
