\chapter*{Введение}
\addcontentsline{toc}{chapter}{Введение}

\section*{Расположение инвариантов цикла}

Как известно, время исполнения программы распределено неравномерно по коду программы.
Подавляющая часть этого времени, по некоторым оценкам порядка 90\%, проводится в малой части кода программы (10\% соответственно) \cite{Aho_Ullman_focs}.
Такие участки зачастую являются телом некоторого цикла.

Рассмотрим инварианты цикла - результат исполнения инструкции не обладающей сторонними эффектами, который не зависит от итерации цикла.
Для значений существует множество расположений -- базовых блоков графа потока управления, в которых расположены инструкции, порождающие эти значения.
Вследствие свойств инвариантов цикла, для них мощность такого множества больше, чем для остальных переменных в цикле.
Так, например, в отличие от индуктивной переменной, инвариант может быть вынесен из цикла.

Как уже было сказано выше, скорость исполнения тела цикла является определяющим фактором скорости исполнения программы в целом.
Из этого утверждения становится ясна важность минимизации времени исполнения блоков цикла.
Субоптимальное расположение инвариантов цикла может негативно влиять на время исполнения программы, так как увеличит число инструкций, которое необходимо исполнить.

\section*{Мотивационный пример}

На первый взгляд, достаточно вынести инварианты из цикла, расположив их перед входом в цикл \cite{Aho_Sethi_Ullman_9_1_7}.
Эта трансформация корректна, так как инвариант будет определен в точке его использования.
Однако, такой подход не всегда обеспечивает минимальную суммарную частоту исполнения блоков расположения инварианта.

Примером случая, когда вынос инвариантов негативно влияет на производительность, является функция \texttt{S\_regmatch} из бенчмарка\\\texttt{perlbench\_r} который является частью набора бенчмарков SPEC CPU\textsuperscript{\tiny\textregistered} 2017.

Данная функция имеет структуру, представленную на схеме \ref{fig:S_regmatch_concept}.
\begin{figure}
    \centering
    \includegraphics[width=\linewidth]{images/S_regmatch_concept.pdf}
    \caption{Схема потока управления в функции \texttt{S\_regmatch}}
    \label{fig:S_regmatch_concept}
\end{figure}

Функция состоит из внешнего цикла:
\\\texttt{body} $\rightarrow$ \texttt{switch} $\rightarrow$ \texttt{case ...} $\rightarrow$\texttt{loop condition} $\rightarrow$ \texttt{body}

Внутри цикла находится конструкция \texttt{switch} с большим количеством вариантов.
Большинство вариантов передают поток управления обратно к условию цикла, но некоторые передают его внутрь цикла, в базовый блок, завершающийся конструкцией ветвления.
Таким образом, образуется внутренний цикл:
\\\texttt{switch} $\rightarrow$ \texttt{case i} $\rightarrow$ \texttt{switch}

Вследствие данной структуры графа потока управления, тело внешнего цикла посещается чаще, чем тело внутреннего цикла в ходе исполнения программы.
Таким образом, если некоторый инвариант внутреннего цикла, который не является инвариантом внешнего цикла, будет вынесен, он окажется в теле внешнего цикла и число исполнений этой инструкции увеличится.

\section*{Постановка задачи}

На приведенном выше примере, Clang (компилятор языка С, использующий компиляторную инфраструктуру LLVM), показывает субоптимальный результат.
Среднее число инструкций на итерацию цикла больше, чем у компилятора GCC.

Это показывает на возможность улучшения алгоритма расположения инвариантов цикла в компиляторной инфраструктуре LLVM, что и является целью данной работы.

Для выполнения этой цели были поставлены следующие задачи:
\begin{itemize}
    \item Исследовать алгоритм расположения инвариантов цикла в компиляторной инфраструктуре LLVM.
    \item Разработать модификации алгоритма.
    \item Доказать, что модификации алгоритма приводят к более оптимальному расположению инвариантов цикла.
    \item Реализовать предложенные улучшения в компиляторной инфраструктуре LLVM.
    \item Продемонстрировать эффективность улучшений, измерив изменение производительности компилируемых программ.
\end{itemize}

\newpage
