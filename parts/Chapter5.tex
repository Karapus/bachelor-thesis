\chapter{Заключение}
\label{sec:Chapter5} \index{Chapter5}

Как отмечено в главе \ref{sec:Chapter1}, эффективность алгоритма расположения инвариантов цикла в компиляторной инфраструктуре LLVM нуждалась в улучшении до применения предлагаемых в работе изменений.
В ходе работы работе был проведен общий анализ алгоритма расположения инвариантов цикла, описанный в главе \ref{sec:Chapter2}, и анализ его применения к функции \texttt{S\_regmatch}, на которой наблюдался значительный разрыв в производительности (главы \ref{sec:Chapter1} и \ref{sec:Chapter3}).
В результате этого анализа, было показано, что причиной неэффективности являются недостатки в алгоритме пропагации инвариантов в тело цикла.

Были разработаны улучшения алгоритма пропагации инвариантов в тело цикла, описанные в главе \ref{sec:Chapter4}.
Для улучшенного алгоритма, была доказана оптимальность получаемого расположения инвариантов.

Алгоритм был реализован в компиляторной инфраструктуре LLVM.
Асимптотика времени исполнения алгоритма является в среднем линейной, а в худшем случае квадратичной, от числа инструкций функции, как показано в главе \ref{sec:Chapter5}.

Проведен анализ производительности алгоритма на наборе бенчмарков SPEC CPU\textsuperscript{\tiny\textregistered} 2017 и наборе бенчмарков из коллекции тестов LLVM.
Этот анализ показал, что предложенные улучшения алгоритма:
\begin{itemize}
    \item Значительно увеличивают производительность некоторых приложений --
        До $12.5\%$ и $2.5\%$ по времени исполнения и числу исполненных инструкций соответственно.
    \item Обеспечивают средний прирост производительности на используемом наборе бенчмарков.
\end{itemize}

Предлагаемые изменения алгоритма пропагации инвариантов цикла были включены во внутреннюю поставку компилятора компании Синтакор.

\newpage
