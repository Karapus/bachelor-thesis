\chapter{Введение}
\label{sec:Chapter0} \index{Chapter0}

\section{Циклы}

Как известно, время исполнения программы распредеоено неравномерно по коду программы.
Подавляющяя часть этого времени, по некоторым оценкам порядка 90\%, проводится в малой части кода программы (10\% соответственно) \cite{Aho_Ullman_focs}.
Такие участки зачастую являются телом некоторого цикла, где под циклом подразумевается максимальное подмножество вершин графа потока управления, такое что:
\begin{itemize}
    \item Индуцированный подграф сильно связан.
    \item Все ребра, входящие в подмножество, направлены в единственную вершину - заголовок.
\end{itemize}

Для дальнейшего описания обработки циклов в компиляторной инфраструктуре, требуются следующие определения:
\begin{itemize}
    \item Входной блок (англ. entering block) - базовый блок, не принадлежащий циклу и имеющий ребро, ведущее в заголовок цикла.
    \item Предзаголовок цикла (англ. preheader) - единственный входной блок цикла.
    \item Выходящее ребро (англ. exiting edge) - ребро, направленое из блока внутри цикла в блок вне цикла.
    \item Блок выхода (англ. exit block) - блок, в который ведет выходящее ребро.
    \item Инвариант цикла - инструкция, результат исполнения которой не зависит от итерации цикла и которая не обладает стороними эффектами.
\end{itemize}

\section{Упрощенное представление циклов}

В связи с вышесказанным о распределении времени исполнения по коду программы, трансформации, связанные с циклами, предоставляют значительную возможность для оптимизирующего компилятора уменьшить время исполнения программы.
Для упрощения описания таких трансформаций в компиляторной инфраструктуре применяются различные внутренние представления. Основное такое представление это упрощенное представление цикла (англ. Loop Simplify Form). Оно обладает следующими свойствами:
\begin{itemize}
    \item У цикла есть предзаголовок.
    \item У цикла единственное обратное ребро.
    \item Все ребра, входящие в блоки выхода, выходят из цикла. Таким образом все блоки выхода доминируются заголовком цикла.
\end{itemize}
Остальные представления строятся на основе упрощенного представления циклов.
\newpage
