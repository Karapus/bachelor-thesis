\chapter*{Заключение}
\addcontentsline{toc}{chapter}{Заключение}

Как отмечено во введении, алгоритм расположения инвариантов цикла в компиляторной инфраструктуре LLVM был недостаточно эффективен.
Была поставлена задача улучшения этого алгоритма.

Задача решена в полном объеме, а именно:
\begin{itemize}
    \item В ходе работы работе был проведен общий анализ алгоритма расположения инвариантов цикла, описанный в главе \ref{sec:Chapter2}, и анализ его применения к функции \texttt{S\_regmatch}, на которой наблюдался значительный разрыв в производительности (глава \ref{sec:Chapter3}). В результате этого анализа, было показано, что причиной неэффективности являются недостатки в алгоритме пропагации инвариантов в тело цикла.
    \item Были разработаны улучшения алгоритма пропагации инвариантов в тело цикла -- пропагация инвариантов с использованием в $\varphi$ узлах и пропагация инвариантов во все доминируемые базовые блоки, описанные в главе \ref{sec:Chapter3}.
    \item Для улучшенного алгоритма, была доказана оптимальность получаемого расположения инвариантов.
    \item Алгоритм был реализован в компиляторной инфраструктуре LLVM.
        Асимптотика времени исполнения алгоритма является в среднем линейной, а в худшем случае квадратичной, от числа инструкций функции, как показано в главе \ref{sec:Chapter4}.
    \item Проведен анализ производительности алгоритма на наборе бенчмарков SPEC CPU\textsuperscript{\tiny\textregistered} 2017 и наборе бенчмарков из коллекции тестов LLVM.
        Этот анализ показал, что предложенные улучшения алгоритма:
        \begin{itemize}
            \item Значительно увеличивают производительность некоторых приложений --
                До $12.5\%$ и $2.5\%$ по времени исполнения и числу исполненных инструкций соответственно.
            \item Обеспечивают средний прирост производительности на используемом наборе бенчмарков.
        \end{itemize}
\end{itemize}

Предлагаемые изменения алгоритма пропагации инвариантов цикла были включены во внутреннюю поставку компилятора компании Синтакор.

\newpage
